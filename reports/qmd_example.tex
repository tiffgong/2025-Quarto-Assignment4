% Options for packages loaded elsewhere
\PassOptionsToPackage{unicode}{hyperref}
\PassOptionsToPackage{hyphens}{url}
\PassOptionsToPackage{dvipsnames,svgnames,x11names}{xcolor}
%
\documentclass[
  letterpaper,
  DIV=11,
  numbers=noendperiod]{scrartcl}

\usepackage{amsmath,amssymb}
\usepackage{iftex}
\ifPDFTeX
  \usepackage[T1]{fontenc}
  \usepackage[utf8]{inputenc}
  \usepackage{textcomp} % provide euro and other symbols
\else % if luatex or xetex
  \usepackage{unicode-math}
  \defaultfontfeatures{Scale=MatchLowercase}
  \defaultfontfeatures[\rmfamily]{Ligatures=TeX,Scale=1}
\fi
\usepackage{lmodern}
\ifPDFTeX\else  
    % xetex/luatex font selection
\fi
% Use upquote if available, for straight quotes in verbatim environments
\IfFileExists{upquote.sty}{\usepackage{upquote}}{}
\IfFileExists{microtype.sty}{% use microtype if available
  \usepackage[]{microtype}
  \UseMicrotypeSet[protrusion]{basicmath} % disable protrusion for tt fonts
}{}
\makeatletter
\@ifundefined{KOMAClassName}{% if non-KOMA class
  \IfFileExists{parskip.sty}{%
    \usepackage{parskip}
  }{% else
    \setlength{\parindent}{0pt}
    \setlength{\parskip}{6pt plus 2pt minus 1pt}}
}{% if KOMA class
  \KOMAoptions{parskip=half}}
\makeatother
\usepackage{xcolor}
\setlength{\emergencystretch}{3em} % prevent overfull lines
\setcounter{secnumdepth}{-\maxdimen} % remove section numbering
% Make \paragraph and \subparagraph free-standing
\makeatletter
\ifx\paragraph\undefined\else
  \let\oldparagraph\paragraph
  \renewcommand{\paragraph}{
    \@ifstar
      \xxxParagraphStar
      \xxxParagraphNoStar
  }
  \newcommand{\xxxParagraphStar}[1]{\oldparagraph*{#1}\mbox{}}
  \newcommand{\xxxParagraphNoStar}[1]{\oldparagraph{#1}\mbox{}}
\fi
\ifx\subparagraph\undefined\else
  \let\oldsubparagraph\subparagraph
  \renewcommand{\subparagraph}{
    \@ifstar
      \xxxSubParagraphStar
      \xxxSubParagraphNoStar
  }
  \newcommand{\xxxSubParagraphStar}[1]{\oldsubparagraph*{#1}\mbox{}}
  \newcommand{\xxxSubParagraphNoStar}[1]{\oldsubparagraph{#1}\mbox{}}
\fi
\makeatother


\providecommand{\tightlist}{%
  \setlength{\itemsep}{0pt}\setlength{\parskip}{0pt}}\usepackage{longtable,booktabs,array}
\usepackage{calc} % for calculating minipage widths
% Correct order of tables after \paragraph or \subparagraph
\usepackage{etoolbox}
\makeatletter
\patchcmd\longtable{\par}{\if@noskipsec\mbox{}\fi\par}{}{}
\makeatother
% Allow footnotes in longtable head/foot
\IfFileExists{footnotehyper.sty}{\usepackage{footnotehyper}}{\usepackage{footnote}}
\makesavenoteenv{longtable}
\usepackage{graphicx}
\makeatletter
\newsavebox\pandoc@box
\newcommand*\pandocbounded[1]{% scales image to fit in text height/width
  \sbox\pandoc@box{#1}%
  \Gscale@div\@tempa{\textheight}{\dimexpr\ht\pandoc@box+\dp\pandoc@box\relax}%
  \Gscale@div\@tempb{\linewidth}{\wd\pandoc@box}%
  \ifdim\@tempb\p@<\@tempa\p@\let\@tempa\@tempb\fi% select the smaller of both
  \ifdim\@tempa\p@<\p@\scalebox{\@tempa}{\usebox\pandoc@box}%
  \else\usebox{\pandoc@box}%
  \fi%
}
% Set default figure placement to htbp
\def\fps@figure{htbp}
\makeatother

\KOMAoption{captions}{tableheading}
\makeatletter
\@ifpackageloaded{caption}{}{\usepackage{caption}}
\AtBeginDocument{%
\ifdefined\contentsname
  \renewcommand*\contentsname{Table of contents}
\else
  \newcommand\contentsname{Table of contents}
\fi
\ifdefined\listfigurename
  \renewcommand*\listfigurename{List of Figures}
\else
  \newcommand\listfigurename{List of Figures}
\fi
\ifdefined\listtablename
  \renewcommand*\listtablename{List of Tables}
\else
  \newcommand\listtablename{List of Tables}
\fi
\ifdefined\figurename
  \renewcommand*\figurename{Figure}
\else
  \newcommand\figurename{Figure}
\fi
\ifdefined\tablename
  \renewcommand*\tablename{Table}
\else
  \newcommand\tablename{Table}
\fi
}
\@ifpackageloaded{float}{}{\usepackage{float}}
\floatstyle{ruled}
\@ifundefined{c@chapter}{\newfloat{codelisting}{h}{lop}}{\newfloat{codelisting}{h}{lop}[chapter]}
\floatname{codelisting}{Listing}
\newcommand*\listoflistings{\listof{codelisting}{List of Listings}}
\makeatother
\makeatletter
\makeatother
\makeatletter
\@ifpackageloaded{caption}{}{\usepackage{caption}}
\@ifpackageloaded{subcaption}{}{\usepackage{subcaption}}
\makeatother

\usepackage{bookmark}

\IfFileExists{xurl.sty}{\usepackage{xurl}}{} % add URL line breaks if available
\urlstyle{same} % disable monospaced font for URLs
\hypersetup{
  pdftitle={DSCI 310: Historical Horse Population in Canada},
  pdfauthor={Tiffany Timbers \& Jordan Bourak},
  colorlinks=true,
  linkcolor={blue},
  filecolor={Maroon},
  citecolor={Blue},
  urlcolor={Blue},
  pdfcreator={LaTeX via pandoc}}


\title{DSCI 310: Historical Horse Population in Canada}
\author{Tiffany Timbers \& Jordan Bourak}
\date{}

\begin{document}
\maketitle

\renewcommand*\contentsname{Table of contents}
{
\hypersetup{linkcolor=}
\setcounter{tocdepth}{3}
\tableofcontents
}

\subsection{Aim}\label{aim}

This project explores the historical population of horses in Canada
between 1906 and 1972 for each province.

\subsection{Data}\label{data}

Horse population data were sourced from the
\href{http://open.canada.ca/en/open-data}{Government of Canada's Open
Data website} (Government of Canada, 2017a and Government of Canada,
2017b).

\subsection{Methods}\label{methods}

The R programming language (R Core Team, 2019) and the following R
packages were used to perform the analysis: knitr (Xie 2014), tidyverse
(Wickham 2017), and Quarto (Allaire et al 2022). \emph{Note: this report
is adapted from Timbers (2020).}

\subsection{Results}\label{results}

\begin{figure}

\centering{

\includegraphics[width=0.8\linewidth,height=\textheight,keepaspectratio]{../results/horse_pops_plot.png}

}

\caption{\label{fig-horse-populations}Horse populations for all
provinces in Canada from 1906 - 1972.}

\end{figure}%

We can see from Figure~\ref{fig-horse-populations} that Ontario,
Saskatchewan and Alberta have had the highest horse populations in
Canada. All provinces have had a decline in horse populations since
1940. This is likely due to the rebound of the Canadian automotive
industry after the Great Depression and the Second World War. An
interesting follow-up visualisation would be car sales per year for each
Province over the time period visualised above to further support this
hypothesis.

Suppose we were interested in looking in more closely at the province
with the highest spread (in terms of standard deviation) of horse
populations. We present the standard deviations in
Table~\ref{tbl-horse-sd}.

\begin{longtable}[]{@{}lr@{}}

\caption{\label{tbl-horse-sd}Standard deviation of historical
(1906-1972) horse populations for each Canadian province.}

\tabularnewline

\toprule\noalign{}
Province & Std \\
\midrule\noalign{}
\endhead
\bottomrule\noalign{}
\endlastfoot
Saskatchewan & 377265.58 \\
Ontario & 266435.32 \\
Alberta & 266063.19 \\
Manitoba & 122403.87 \\
Quebec & 111411.10 \\
New Brunswick & 22019.49 \\
Nova Scotia & 19879.25 \\
British Columbia & 14945.66 \\
P.E.I. & 11355.75 \\

\end{longtable}

Note that we define standard deviation (of a sample) as

\[s = \sqrt{\frac{\sum_{i=1}^N (x_i - \overline{x})^2}{N-1} }\]

Additionally, note that in Table~\ref{tbl-horse-sd} we consider the
sample standard deviation of the number of horses during the same time
span as Figure~\ref{fig-horse-populations}.

\begin{figure}

\centering{

\includegraphics[width=0.65\linewidth,height=\textheight,keepaspectratio]{../results/horse_pop_plot_largest_sd.png}

}

\caption{\label{fig-largest-sd}Horse populations for the province with
the largest standard deviation.}

\end{figure}%

In Figure~\ref{fig-largest-sd} we zoom in and look at the province of
Saskatchewan, which had the largest spread of values in terms of
standard deviation.

\subsection{References}\label{references}




\end{document}
